\documentclass[cn, 11pt, fancy, hide]{elegantbook}

\usepackage{natbib}
\bibliographystyle{apalike}

\hypersetup{
  pdfcreator={LaTeX via pandoc}}


\usepackage{longtable,booktabs}



\setlength{\emergencystretch}{3em}  % prevent overfull lines
\providecommand{\tightlist}{%
  \setlength{\itemsep}{0pt}\setlength{\parskip}{0pt}}

\setcounter{secnumdepth}{5}

%%% Use protect on footnotes to avoid problems with footnotes in titles
\let\rmarkdownfootnote\footnote%
\def\footnote{\protect\rmarkdownfootnote}

  \title{大数据背景下精准科研信息服务}

\subtitle{2020年版}

  \author{王敏杰}

  \date{2020-04-09}

% logo 图案
  \logo{figure/logo.png}
% 封面图片
  \cover{figure/cover.jpg}
% 版本号
  \version{0.1}
% 机构名
% 引用格言
  \extrainfo{Victory won't come to us unless we go to it. --- M. Moore}
% 导言区 preamble
\usepackage{framed,color}
\definecolor{shadecolor}{RGB}{248,248,248}

\frontmatter
\usepackage{booktabs}
\usepackage{longtable}
\usepackage{array}
\usepackage{multirow}
\usepackage{wrapfig}
\usepackage{float}
\usepackage{colortbl}
\usepackage{pdflscape}
\usepackage{tabu}
\usepackage{threeparttable}
\usepackage{threeparttablex}
\usepackage[normalem]{ulem}
\usepackage{makecell}
\usepackage{xcolor}

\begin{document}
% 封面
\maketitle
% 插入 before_body.tex

% 目录
{
\setcounter{tocdepth}{2}
\tableofcontents
}
% 表目录
% 图目录
% 书籍主体部分
\mainmatter

\hypersetup{pageanchor=true}

\hypertarget{author}{%
\chapter*{作者简介}\label{author}}
\addcontentsline{toc}{chapter}{作者简介}

王敏杰,四川师范大学研究生公选课《数据科学中的R语言》授课老师,西南交通大学量子物理学博士,爱好数据科学,喜欢用R和stan编程,
联系方式 \href{mailto:38552109@qq.com}{\nolinkurl{38552109@qq.com}}

\hypertarget{preface}{%
\chapter{前言}\label{preface}}

\hypertarget{ux8fdbux5ea6ux8868}{%
\section{进度表}\label{ux8fdbux5ea6ux8868}}

\begin{itemize}
\tightlist
\item
  文献调研(3月底完成)
\item
  数据获取(4月中旬完成)
\item
  数学分析和模型评估(5月中旬完成)
\item
  可视化(6月初完成)
\item
  报告初稿(6月底完成)
\item
  研讨会(待定)
\item
  正式稿发布(7月初)
\end{itemize}

\hypertarget{ux9700ux8981ux7684ux914dux5957}{%
\section{需要的配套}\label{ux9700ux8981ux7684ux914dux5957}}

\begin{itemize}
\tightlist
\item
  需要一名学生,协助完成数据收集和整理工作(图书馆提供劳务费)
\end{itemize}

\hypertarget{ux5173ux4e8eux672cux6587ux6863}{%
\section{关于本文档}\label{ux5173ux4e8eux672cux6587ux6863}}

本报告使用R和stan语言完成,数据和代码存放在GitHub仓库\url{https://github.com/perlatex/ElegantBookdown4IS},欢迎批评指正。

\hypertarget{bigdata}{%
\chapter{川师大数据}\label{bigdata}}

\hypertarget{ux5168ux666fux5bf9ux6bd4}{%
\section{全景对比}\label{ux5168ux666fux5bf9ux6bd4}}

横向比较top30所师范类高校的学科发展情况

\begin{itemize}
\item
  全景大图
  高亮的(包括川师在内的四个学校)类似R4DS中的eda\_covid2019
  吸取\href{https://github.com/kjhealy/covid}{Kieran Healy大神的配色方案}
\item
  学科小图
\end{itemize}

\hypertarget{ux5b66ux79d1ux5bf9ux6bd4}{%
\section{学科对比}\label{ux5b66ux79d1ux5bf9ux6bd4}}

分面各校,高亮川师

\hypertarget{predict}{%
\chapter{学科预测}\label{predict}}

本章的主要工作是,计算并预测川师未来三年进入双一流学科的概率。
可能一点意义也没有

\hypertarget{ux7edfux8ba1ux65b9ux6cd5}{%
\section{统计方法}\label{ux7edfux8ba1ux65b9ux6cd5}}

\begin{itemize}
\tightlist
\item
  贝叶斯数据分析
\item
  模型不好怎么办?
\item
  log10 scale
\end{itemize}

关于模型,学科的发展和很多方面都有关系,因此建立一个完全正确的模型是不可能的。正如英国统计学家George E. P. Box所说,所有模型都是错的,但其中有些是有用的。
所以与其去建立复杂的模型,并给解释带来更多困扰,不如就从最简单的出发。

\hypertarget{ux6570ux5b66ux5b66ux79d1}{%
\section{数学学科}\label{ux6570ux5b66ux5b66ux79d1}}

\hypertarget{ux7269ux7406ux5b66ux79d1}{%
\section{物理学科}\label{ux7269ux7406ux5b66ux79d1}}

\hypertarget{ux6570ux636e}{%
\subsection{数据}\label{ux6570ux636e}}

\begin{center}\includegraphics[width=1\linewidth]{ElegantBookdown_files/figure-latex/unnamed-chunk-3-1} \end{center}

\hypertarget{ux5148ux9a8cux6982ux7387}{%
\subsection{先验概率}\label{ux5148ux9a8cux6982ux7387}}

\begin{verbatim}
#>  Family: gaussian 
#>   Links: mu = identity; sigma = identity 
#> Formula: n_cited ~ 1 + year 
#>    Data: d2 (Number of observations: 20) 
#> Samples: 4 chains, each with iter = 41000; warmup = 40000; thin = 1;
#>          total post-warmup samples = 4000
#> 
#> Population-Level Effects: 
#>           Estimate Est.Error l-95% CI u-95% CI Rhat Bulk_ESS Tail_ESS
#> Intercept -2266.03    516.07 -3265.95 -1229.33 1.00     2670     2280
#> year          2.63      0.26     2.12     3.13 1.00     2671     2294
#> 
#> Family Specific Parameters: 
#>       Estimate Est.Error l-95% CI u-95% CI Rhat Bulk_ESS Tail_ESS
#> sigma     6.51      1.20     4.68     9.34 1.00     2543     2096
#> 
#> Samples were drawn using sampling(NUTS). For each parameter, Bulk_ESS
#> and Tail_ESS are effective sample size measures, and Rhat is the potential
#> scale reduction factor on split chains (at convergence, Rhat = 1).
\end{verbatim}

\begin{verbatim}
#> # A tibble: 12,000 x 6
#> # Groups:   year, .row [3]
#>     year  .row .chain .iteration .draw .prediction
#>    <dbl> <int>  <int>      <int> <int>       <dbl>
#>  1  2020     1     NA         NA     1       3049.
#>  2  2020     1     NA         NA     2       3052.
#>  3  2020     1     NA         NA     3       3064.
#>  4  2020     1     NA         NA     4       3056.
#>  5  2020     1     NA         NA     5       3044.
#>  6  2020     1     NA         NA     6       3044.
#>  7  2020     1     NA         NA     7       3051.
#>  8  2020     1     NA         NA     8       3056.
#>  9  2020     1     NA         NA     9       3041.
#> 10  2020     1     NA         NA    10       3056.
#> # ... with 11,990 more rows
\end{verbatim}

\begin{verbatim}
#> # A tibble: 3 x 3
#>    year pred_mean prob_above_line
#>   <dbl>     <dbl>           <dbl>
#> 1  2020     3050.           0.911
#> 2  2021     3053.           0.956
#> 3  2022     3055.           0.974
\end{verbatim}

\begin{center}\includegraphics[width=1\linewidth]{ElegantBookdown_files/figure-latex/unnamed-chunk-13-1} \end{center}

\hypertarget{ux5316ux5b66ux5b66ux79d1}{%
\section{化学学科}\label{ux5316ux5b66ux5b66ux79d1}}

\hypertarget{ux5de5ux7a0bux5b66ux79d1}{%
\section{工程学科}\label{ux5de5ux7a0bux5b66ux79d1}}

\hypertarget{ux8ba1ux7b97ux673aux5b66ux79d1}{%
\section{计算机学科}\label{ux8ba1ux7b97ux673aux5b66ux79d1}}

具体参考 \url{https://mc-stan.org/docs/2_22/stan-users-guide/prediction-forecasting-and-backcasting.html}

\begin{verbatim}
#> Inference for Stan model: simple.
#> 4 chains, each with iter=41000; warmup=40000; thin=1; 
#> post-warmup draws per chain=1000, total post-warmup draws=4000.
#> 
#>              mean se_mean     sd     2.5%      25%      50%      75%    97.5%
#> alpha    -2007.74    3.83 102.57 -2211.51 -2074.88 -2008.05 -1941.31 -1800.92
#> beta         2.50    0.00   0.05     2.40     2.47     2.50     2.54     2.60
#> sigma        6.35    0.03   1.07     4.65     5.59     6.19     6.97     8.79
#> new_y[1]  3048.54    0.11   6.59  3035.59  3044.20  3048.30  3052.82  3061.71
#> new_y[2]  3051.05    0.11   6.74  3037.94  3046.68  3051.01  3055.41  3064.60
#> new_y[3]  3053.55    0.10   6.63  3040.36  3049.32  3053.61  3058.00  3066.21
#> lp__       -63.35    0.05   1.26   -66.57   -63.94   -63.03   -62.44   -61.94
#>          n_eff Rhat
#> alpha      718 1.01
#> beta       717 1.01
#> sigma     1148 1.00
#> new_y[1]  3784 1.00
#> new_y[2]  3863 1.00
#> new_y[3]  4034 1.00
#> lp__       755 1.00
#> 
#> Samples were drawn using NUTS(diag_e) at Thu Apr 09 18:56:26 2020.
#> For each parameter, n_eff is a crude measure of effective sample size,
#> and Rhat is the potential scale reduction factor on split chains (at 
#> convergence, Rhat=1).
\end{verbatim}

\hypertarget{ux9884ux6d4b}{%
\subsection{预测}\label{ux9884ux6d4b}}

\begin{verbatim}
#> # A tibble: 12,000 x 5
#> # Groups:   condition [3]
#>    condition new_y .chain .iteration .draw
#>        <int> <dbl>  <int>      <int> <int>
#>  1         1 3048.      1          1     1
#>  2         1 3047.      1          2     2
#>  3         1 3049.      1          3     3
#>  4         1 3044.      1          4     4
#>  5         1 3058.      1          5     5
#>  6         1 3041.      1          6     6
#>  7         1 3054.      1          7     7
#>  8         1 3051.      1          8     8
#>  9         1 3050.      1          9     9
#> 10         1 3050.      1         10    10
#> # ... with 11,990 more rows
\end{verbatim}

\begin{verbatim}
#> # A tibble: 12,000 x 6
#>    condition new_y .chain .iteration .draw  year
#>        <int> <dbl>  <int>      <int> <int> <dbl>
#>  1         1 3048.      1          1     1  2020
#>  2         1 3047.      1          2     2  2020
#>  3         1 3049.      1          3     3  2020
#>  4         1 3044.      1          4     4  2020
#>  5         1 3058.      1          5     5  2020
#>  6         1 3041.      1          6     6  2020
#>  7         1 3054.      1          7     7  2020
#>  8         1 3051.      1          8     8  2020
#>  9         1 3050.      1          9     9  2020
#> 10         1 3050.      1         10    10  2020
#> # ... with 11,990 more rows
\end{verbatim}

\begin{verbatim}
#> # A tibble: 3 x 3
#>    year pred_mean prob_above_line
#>   <dbl>     <dbl>           <dbl>
#> 1  2020     3049.           0.911
#> 2  2021     3051.           0.950
#> 3  2022     3054.           0.978
\end{verbatim}

\begin{center}\includegraphics[width=1\linewidth]{ElegantBookdown_files/figure-latex/unnamed-chunk-27-1} \end{center}

\begin{center}\includegraphics[width=1\linewidth]{ElegantBookdown_files/figure-latex/unnamed-chunk-28-1} \end{center}

\begin{center}\includegraphics[width=1\linewidth]{ElegantBookdown_files/figure-latex/unnamed-chunk-29-1} \end{center}

\hypertarget{ux5176ux4ed6ux5b66ux79d1}{%
\section{其他学科}\label{ux5176ux4ed6ux5b66ux79d1}}

如果需要了解其他学科的信息,请联系本文作者\footnote{\href{mailto:38552109@qq.com}{\nolinkurl{38552109@qq.com}}}

\hypertarget{collegecontr}{%
\chapter{学院对学科的贡献}\label{collegecontr}}

\hypertarget{ux7814ux7a76ux89c4ux6a21ux8d21ux732eux5206ux6790}{%
\section{研究规模贡献分析}\label{ux7814ux7a76ux89c4ux6a21ux8d21ux732eux5206ux6790}}

数据不准确,避免引起歧义。暂时不开展

\hypertarget{ux5b66ux672fux5f71ux54cdux529bux8d21ux732eux5206ux6790}{%
\section{学术影响力贡献分析}\label{ux5b66ux672fux5f71ux54cdux529bux8d21ux732eux5206ux6790}}

数据不准确,避免引起歧义。暂时不开展

\hypertarget{jcr}{%
\chapter{选刊倾向与期刊推荐}\label{jcr}}

\hypertarget{ux5404ux5b66ux79d1ux8bbaux6587ux5728ux5404ux7b49ux7ea7ux671fux520aux4e0aux7684ux5206ux5e03}{%
\section{各学科论文在各等级期刊上的分布}\label{ux5404ux5b66ux79d1ux8bbaux6587ux5728ux5404ux7b49ux7ea7ux671fux520aux4e0aux7684ux5206ux5e03}}

数据不准确,避免引起歧义。暂时不开展

\hypertarget{ux671fux520aux63a8ux8350}{%
\section{期刊推荐}\label{ux671fux520aux63a8ux8350}}

数据不准确,避免引起歧义。暂时不开展

\cleardoublepage

\hypertarget{appendix-ux9644ux5f55}{%
\appendix}


\hypertarget{sound}{%
\chapter{统计口径}\label{sound}}

\hypertarget{ux5b66ux79d1ux5206ux7c7bux4ee5ux53caux5404ux5b66ux79d1ux8fdbux5165esiux7684ux9608ux503c}{%
\section{学科分类以及各学科进入ESI的阈值}\label{ux5b66ux79d1ux5206ux7c7bux4ee5ux53caux5404ux5b66ux79d1ux8fdbux5165esiux7684ux9608ux503c}}

ESI 学科分类
一种较为宽泛的学科分类模式。ESI 学科分类模式基于期刊分类,由自然科学与
社会科学的 22 个学科构成。艺术与人文期刊没有被包含。每一本期刊只被划分
至 22 个 ESI 学科中的一个,没有重叠的学科设置使得分析变得更为简单。
被归类为跨学科学科(Multidisciplinary field)的Science、Nature与PNAS期刊,会被按照各篇文章的参考文献(reference)与引用文献(citation),重新为每篇文章单独分类,但每篇文章仍只会被分类到一个学科。

\begin{table}

\caption{\label{tab:unnamed-chunk-31}ESI学科分类以及各学科进入ESI的阈值(2020年3月数据)}
\centering
\begin{tabular}[t]{>{\bfseries\raggedright\arraybackslash}p{8em}lr}
\toprule
category & discipline & threshold202003\\
\rowcolor{gray!15}
\midrule
 & 计算机科学(Computer Science) & 1692\\

\rowcolor{gray!15}
 & 工程科学(Engineering) & 5079\\

\rowcolor{gray!15}
\multirow{-3}{8em}{\raggedright\arraybackslash 工学(3)} & 材料科学(Materials Sciences) & 5981\\

 & 生物与生化(Biology \& Biochemistry) & 1855\\

 & 环境/生态学(Environment/Ecology) & 2837\\

 & 微生物学(Microbiology) & 3549\\

\multirow{-4}{8em}{\raggedright\arraybackslash 生命科学(4)} & 分子生物与遗传学(Molecular Biology \& Genetics) & 1876\\

\rowcolor{gray!15}
 & 一般社会科学(Social Sciences, General) & 3319\\

\rowcolor{gray!15}
\multirow{-2}{8em}{\raggedright\arraybackslash 社会科学(2)} & 经济与商学(Economics \& Business) & 4795\\

 & 化学(Chemistry) & 3844\\

 & 地球科学(Geosciences) & 3918\\

 & 数学(Mathematics) & 3620\\

 & 物理学(Physics) & 4421\\

\multirow{-5}{8em}{\raggedright\arraybackslash 理学(5)} & 空间科学(Space Science) & 10243\\

\rowcolor{gray!15}
 & 农业科学(Agricultural Sciences) & 2087\\

\rowcolor{gray!15}
\multirow{-2}{8em}{\raggedright\arraybackslash 农学(2)} & 植物与动物科学(Plant \& Animal Science) & 4959\\

 & 临床医学(Clinical Medicine) & 2864\\

 & 免疫学(Immunology) & 14029\\

 & 神经科学与行为(Neuroscience \& Behavior) & 2236\\

 & 药理学与毒物学(Pharmacology \& Toxicology) & 3464\\

\multirow{-5}{8em}{\raggedright\arraybackslash 医学(5)} & 精神病学/心理学(Psychology/Psychiatry) & 1142\\

\rowcolor{gray!15}
其他(1) & 多学科(Multidisciplinary) & 27851\\
\bottomrule
\end{tabular}
\end{table}

\hypertarget{ux6570ux636eux6765ux6e90}{%
\section{数据来源}\label{ux6570ux636eux6765ux6e90}}

\begin{itemize}
\tightlist
\item
  用ESI 不用wos
\item
  2010 - 2019 十年, 6个学科(数学,物理, 化学,工程, 计算机)
\item
  获取下载地址

  \begin{itemize}
  \tightlist
  \item
    链接1,检索学校历年发文量的(\url{https://incites.clarivate.com/zh/\#/explore/0/subject})
  \item
    链接2,近期进入ESI学科的阈值(\url{https://esi.clarivate.com/ThresholdsAction.action})
  \end{itemize}
\end{itemize}

\hypertarget{ux83b7ux53d6ux65b9ux6cd5}{%
\section{获取方法}\label{ux83b7ux53d6ux65b9ux6cd5}}

整理的raw-data 可以在这里找到
\url{https://github.com/perlatex/ElegantBookdown4IS/tree/master/data}

\hypertarget{ux5b66ux6821ux5217ux8868}{%
\section{学校列表}\label{ux5b66ux6821ux5217ux8868}}

\begin{itemize}
\tightlist
\item
  师范类学校清单(\url{https://www.dxsbb.com/news/1448.html})
\item
  选取依据(top30)(川师25名)
\item
  省份 中文名 英文名
\end{itemize}

\hypertarget{ux7269ux7406ux5b66ux79d1ux6a21ux578bux53c2ux6570}{%
\section{物理学科模型参数}\label{ux7269ux7406ux5b66ux79d1ux6a21ux578bux53c2ux6570}}

\begin{center}\includegraphics[width=1\linewidth]{ElegantBookdown_files/figure-latex/unnamed-chunk-32-1} \end{center}

\begin{center}\includegraphics[width=1\linewidth]{ElegantBookdown_files/figure-latex/unnamed-chunk-33-1} \end{center}

\hypertarget{ux53c2ux8003ux6587ux4ef6}{%
\section{参考文件}\label{ux53c2ux8003ux6587ux4ef6}}

\begin{itemize}
\tightlist
\item
  stan
\item
  ggplot
\item
  tidybayes
\item
  tidyverse
\item
  书籍
\end{itemize}
% 参考文献
\bibliography{book.bib}



% 插入 after_body.tex

\end{document}
