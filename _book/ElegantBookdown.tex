\documentclass[cn, 11pt, fancy, hide]{elegantbook}

\usepackage{natbib}
\bibliographystyle{apalike}

\hypersetup{
  pdfcreator={LaTeX via pandoc}}


\usepackage{longtable,booktabs}



\setlength{\emergencystretch}{3em}  % prevent overfull lines
\providecommand{\tightlist}{%
  \setlength{\itemsep}{0pt}\setlength{\parskip}{0pt}}

\setcounter{secnumdepth}{5}

%%% Use protect on footnotes to avoid problems with footnotes in titles
\let\rmarkdownfootnote\footnote%
\def\footnote{\protect\rmarkdownfootnote}

  \title{大数据背景下精准科研信息服务}

\subtitle{2020年版}

  \author{王敏杰}

  \date{2020-04-06}

% logo 图案
  \logo{figure/logo.png}
% 封面图片
  \cover{figure/cover.jpg}
% 版本号
  \version{0.1}
% 机构名
% 引用格言
  \extrainfo{Victory won't come to us unless we go to it. --- M. Moore}
% 导言区 preamble
\usepackage{framed,color}
\definecolor{shadecolor}{RGB}{248,248,248}

\frontmatter
\usepackage{booktabs}
\usepackage{longtable}
\usepackage{array}
\usepackage{multirow}
\usepackage{wrapfig}
\usepackage{float}
\usepackage{colortbl}
\usepackage{pdflscape}
\usepackage{tabu}
\usepackage{threeparttable}
\usepackage{threeparttablex}
\usepackage[normalem]{ulem}
\usepackage{makecell}
\usepackage{xcolor}

\begin{document}
% 封面
\maketitle
% 插入 before_body.tex

% 目录
{
\setcounter{tocdepth}{2}
\tableofcontents
}
% 表目录
% 图目录
% 书籍主体部分
\mainmatter

\hypersetup{pageanchor=true}

\hypertarget{preface}{%
\chapter{前言}\label{preface}}

\hypertarget{ux8fdbux5ea6ux8868}{%
\section{进度表}\label{ux8fdbux5ea6ux8868}}

\begin{itemize}
\tightlist
\item
  文献调研(3月底完成)
\item
  数据获取(4月中旬完成)
\item
  数学分析和模型评估(5月中旬完成)
\item
  可视化(6月初完成)
\item
  报告初稿(6月底完成)
\item
  研讨会(待定)
\item
  正式稿发布(7月初)
\end{itemize}

\hypertarget{ux9700ux8981ux7684ux914dux5957}{%
\section{需要的配套}\label{ux9700ux8981ux7684ux914dux5957}}

\begin{itemize}
\tightlist
\item
  需要一名学生,协助完成数据收集和整理工作(图书馆提供劳务费)
\end{itemize}

\hypertarget{ux5173ux4e8eux672cux6587ux6863}{%
\section{关于本文档}\label{ux5173ux4e8eux672cux6587ux6863}}

本报告使用R语言完成,数据和代码存放在github仓库\url{https://github.com/perlatex/ElegantBookdown4IS},欢迎批评指正。

\hypertarget{bigdata}{%
\chapter{川师大数据}\label{bigdata}}

\hypertarget{ux5168ux666fux5bf9ux6bd4}{%
\section{全景对比}\label{ux5168ux666fux5bf9ux6bd4}}

横向比较top30所师范类高校的学科发展情况

\hypertarget{ux5b66ux79d1ux5bf9ux6bd4}{%
\section{学科对比}\label{ux5b66ux79d1ux5bf9ux6bd4}}

分面各校,高亮川师

\hypertarget{predict}{%
\chapter{学科预测}\label{predict}}

本章的主要工作是,计算并预测川师未来三年进入双一流学科的概率。
可能一点意义也没有

\hypertarget{ux7edfux8ba1ux65b9ux6cd5}{%
\section{统计方法}\label{ux7edfux8ba1ux65b9ux6cd5}}

\begin{itemize}
\tightlist
\item
  贝叶斯数据分析
\item
  模型不好怎么办?
\item
  log10 scale
\end{itemize}

\hypertarget{ux6570ux5b66ux5b66ux79d1}{%
\section{数学学科}\label{ux6570ux5b66ux5b66ux79d1}}

\hypertarget{ux7269ux7406ux5b66ux79d1}{%
\section{物理学科}\label{ux7269ux7406ux5b66ux79d1}}

\begin{verbatim}
#> # A tibble: 3 x 3
#>   weight pred_height_mean prob_above_line
#>    <dbl>            <dbl>           <dbl>
#> 1     64             172.           0.633
#> 2     68             175.           0.852
#> 3     72             179.           0.960
\end{verbatim}

\begin{center}\includegraphics[width=1\linewidth]{ElegantBookdown_files/figure-latex/unnamed-chunk-10-1} \end{center}

\hypertarget{ux5316ux5b66ux5b66ux79d1}{%
\section{化学学科}\label{ux5316ux5b66ux5b66ux79d1}}

\hypertarget{ux5de5ux7a0bux5b66ux79d1}{%
\section{工程学科}\label{ux5de5ux7a0bux5b66ux79d1}}

\hypertarget{ux8ba1ux7b97ux673aux5b66ux79d1}{%
\section{计算机学科}\label{ux8ba1ux7b97ux673aux5b66ux79d1}}

\hypertarget{ux5176ux4ed6ux5b66ux79d1}{%
\section{其他学科}\label{ux5176ux4ed6ux5b66ux79d1}}

如果需要了解其他学科的信息,请联系本文作者\footnote{\href{mailto:38552109@qq.com}{\nolinkurl{38552109@qq.com}}}

\hypertarget{collegecontr}{%
\chapter{学院对学科的贡献}\label{collegecontr}}

\hypertarget{ux7814ux7a76ux89c4ux6a21ux8d21ux732eux5206ux6790}{%
\section{研究规模贡献分析}\label{ux7814ux7a76ux89c4ux6a21ux8d21ux732eux5206ux6790}}

数据不准确,避免引起歧义。暂时不开展

\hypertarget{ux5b66ux672fux5f71ux54cdux529bux8d21ux732eux5206ux6790}{%
\section{学术影响力贡献分析}\label{ux5b66ux672fux5f71ux54cdux529bux8d21ux732eux5206ux6790}}

数据不准确,避免引起歧义。暂时不开展

\hypertarget{jcr}{%
\chapter{选刊倾向与期刊推荐}\label{jcr}}

\hypertarget{ux5404ux5b66ux79d1ux8bbaux6587ux5728ux5404ux7b49ux7ea7ux671fux520aux4e0aux7684ux5206ux5e03}{%
\section{各学科论文在各等级期刊上的分布}\label{ux5404ux5b66ux79d1ux8bbaux6587ux5728ux5404ux7b49ux7ea7ux671fux520aux4e0aux7684ux5206ux5e03}}

数据不准确,避免引起歧义。暂时不开展

\hypertarget{ux671fux520aux63a8ux8350}{%
\section{期刊推荐}\label{ux671fux520aux63a8ux8350}}

数据不准确,避免引起歧义。暂时不开展

\cleardoublepage

\hypertarget{appendix-ux9644ux5f55}{%
\appendix}


\hypertarget{sound}{%
\chapter{统计口径}\label{sound}}

\hypertarget{ux5b66ux79d1ux5206ux7c7b}{%
\section{学科分类}\label{ux5b66ux79d1ux5206ux7c7b}}

ESI所收录的期刊会被分为22个学科,再依学科进行各项统计。在ESI数据库中,每种期刊只会被分入一个学科;只有被归类为跨学科学科(Multidisciplinary field)的Science、Nature与PNAS期刊,会被按照各篇文章的参考文献(reference)与引用文献(citation),重新为每篇文章单独分类,但每篇文章仍只会被分类到一个学科。

\begin{figure}[H]

{\centering \includegraphics[width=0.9\linewidth]{images/table} 

}

\caption{ESI学科分类}\label{fig:unnamed-chunk-11}
\end{figure}

\hypertarget{ux5404ux5b66ux79d1ux8fdbux5165esiux7684ux9608ux503c}{%
\section{各学科进入ESI的阈值}\label{ux5404ux5b66ux79d1ux8fdbux5165esiux7684ux9608ux503c}}

\begin{verbatim}
#> # A tibble: 22 x 3
#>    学科门类      学科                                            阈值
#>    <chr>         <chr>                                          <dbl>
#>  1 工学(3)     计算机科学(Computer Science)                    1692
#>  2 工学(3)     工程科学(Engineering)                           5079
#>  3 工学(3)     材料科学(Materials Sciences)                    5981
#>  4 生命科学(4) 生物与生化(Biology & Biochemistry)              1855
#>  5 生命科学(4) 环境/生态学(Environment/Ecology)               2837
#>  6 生命科学(4) 微生物学(Microbiology)                          3549
#>  7 生命科学(4) 分子生物与遗传学(Molecular Biology & Genetics)  1876
#>  8 社会科学(2) 一般社会科学(Social Sciences, General)          3319
#>  9 社会科学(2) 经济与商学(Economics & Business)                4795
#> 10 理学(5)     化学(Chemistry)                                 3844
#> # ... with 12 more rows
\end{verbatim}

\begin{table}

\caption{\label{tab:unnamed-chunk-13}各学科进入ESI的阈值(2020年3月数据)}
\centering
\begin{tabular}[t]{>{\bfseries\raggedright\arraybackslash}p{8em}lr}
\toprule
学科门类 & 学科 & 阈值\\
\rowcolor{gray!6}
\midrule
 & 计算机科学(Computer Science) & 1692\\

\rowcolor{gray!6}
 & 工程科学(Engineering) & 5079\\

\rowcolor{gray!6}
\multirow{-3}{8em}{\raggedright\arraybackslash 工学(3)} & 材料科学(Materials Sciences) & 5981\\

 & 生物与生化(Biology \& Biochemistry) & 1855\\

 & 环境/生态学(Environment/Ecology) & 2837\\

 & 微生物学(Microbiology) & 3549\\

\multirow{-4}{8em}{\raggedright\arraybackslash 生命科学(4)} & 分子生物与遗传学(Molecular Biology \& Genetics) & 1876\\

\rowcolor{gray!6}
 & 一般社会科学(Social Sciences, General) & 3319\\

\rowcolor{gray!6}
\multirow{-2}{8em}{\raggedright\arraybackslash 社会科学(2)} & 经济与商学(Economics \& Business) & 4795\\

 & 化学(Chemistry) & 3844\\

 & 地球科学(Geosciences) & 3918\\

 & 数学(Mathematics) & 3620\\

 & 物理学(Physics) & 4421\\

\multirow{-5}{8em}{\raggedright\arraybackslash 理学(5)} & 空间科学(Space Science) & 10243\\

\rowcolor{gray!6}
 & 农业科学(Agricultural Sciences) & 2087\\

\rowcolor{gray!6}
\multirow{-2}{8em}{\raggedright\arraybackslash 农学(2)} & 植物与动物科学(Plant \& Animal Science) & 4959\\

 & 临床医学(Clinical Medicine) & 2864\\

 & 免疫学(Immunology) & 14029\\

 & 神经科学与行为(Neuroscience \& Behavior) & 2236\\

 & 药理学与毒物学(Pharmacology \& Toxicology) & 3464\\

\multirow{-5}{8em}{\raggedright\arraybackslash 医学(5)} & 精神病学/心理学(Psychology/Psychiatry) & 1142\\

\rowcolor{gray!6}
其他(1) & 多学科(Multidisciplinary) & 27851\\
\bottomrule
\end{tabular}
\end{table}

\hypertarget{ux6570ux636eux6765ux6e90}{%
\section{数据来源}\label{ux6570ux636eux6765ux6e90}}

\begin{itemize}
\tightlist
\item
  用ESI 不用wos
\item
  2010 - 2019 十年, 6个学科(数学,物理, 化学,工程, 计算机)
\item
  获取下载地址

  \begin{itemize}
  \tightlist
  \item
    链接1,检索学校历年发文量的
  \item
    链接2,近期进入ESI学科的阈值
  \end{itemize}
\end{itemize}

\hypertarget{ux83b7ux53d6ux65b9ux6cd5}{%
\section{获取方法}\label{ux83b7ux53d6ux65b9ux6cd5}}

整理的raw-data 可以在这里找到

\hypertarget{ux5b66ux6821ux5217ux8868}{%
\section{学校列表}\label{ux5b66ux6821ux5217ux8868}}

\begin{itemize}
\tightlist
\item
  师范类学校清单(\url{https://www.dxsbb.com/news/1448.html})
\item
  选取依据(top30)(川师25名)
\item
  省份 中文名 英文名
\end{itemize}
% 参考文献
\bibliography{book.bib}



% 插入 after_body.tex

\end{document}
